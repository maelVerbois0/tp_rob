%! TEX root = .
\documentclass[a4paper,11pt]{article}

% ====================
% PACKAGES DE BASE
% ====================
\usepackage[utf8]{inputenc}
\usepackage[T1]{fontenc}
\usepackage[french]{babel}
\usepackage{amsmath, amssymb, amsfonts}
\usepackage{graphicx}
\usepackage{geometry}
\usepackage{hyperref}
\usepackage{booktabs}
\usepackage{float}
\usepackage{xcolor}
\usepackage{caption}
\usepackage{subcaption}
\usepackage{siunitx}

% ====================
% MISE EN PAGE
% ====================
\geometry{margin=2.5cm}
\setlength{\parskip}{6pt}
\setlength{\parindent}{0pt}
\hypersetup{
    colorlinks=true,
    linkcolor=blue,
    urlcolor=blue,
    citecolor=blue
}

% ====================
% INFORMATIONS DU DOCUMENT
% ====================
\title{\textbf{Compte rendu du TP4}\\[4pt]
\large Exploitation durable de la forêt}
\author{Groupe : \textit{Maël Verbois, Jules Onée, Reyen Selmi} \\[2pt] Cours : ROB}


% ====================
% DOCUMENT
% ====================
\begin{document}

\maketitle

\section{Compte Rendu}
% ===========================
L’objectif de ce travail pratique est d’étudier un problème d’optimisation combinatoire inspiré de la gestion durable d’une forêt. 
L’exploitation forestière doit être planifiée de manière à préserver deux espèces animales dont les habitats dépendent des parcelles coupées ou conservées. 
Ce TP vise à formuler ce problème sous forme d’un programme linéaire en variables mixtes, puis à en proposer une version quadratique et sa linéarisation totale. 
Nous comparerons ensuite les performances des différentes approches sur plusieurs instances et analyserons l’impact de contraintes supplémentaires sur la structure du problème.

% ===========================
\section{Formulation du modèle initial (P1)}
% ===========================
\subsection{Présentation du modèle}
% → Ici, tu pourras insérer la formulation mathématique de (P1) en environnement align ou cases
% Exemple :


\[
\begin{aligned}
\max_{x,d} \quad & w_1 \sum_{(i,j)\in M\times N} t_{ij}(1 - x_{ij}) + w_2 g l \sum_{(i,j)\in M\times N} (4x_{ij} - d_{ij}) \\
\text{sous contraintes} \quad &
\begin{cases}
d_{ij} \ge \sum_{(k,l)\in A_{ij}} x_{kl} - |A_{ij}|(1 - x_{ij}) \quad \forall (i,j)\in M\times N,\\[8pt]
d_{ij} \ge 0, \quad x_{ij}\in \{0,1\}
\end{cases}
\end{aligned}
\]

\subsection{Interprétation des variables et paramètres}

\begin{itemize}
    \item $M$ et $N$ : dimension de la matrice représentant la forêt
    \item \item $A_{ij}$ : ensemble des indices $(k,l)$ correspondant aux parcelles adjacentes à $(i,j)$ dans la matrice $M \times N$.
    \item $g$ : population attendue de l'espèce $e_2$ par kilomètre de lisière.
    \item $l$ : longueur du côté d’une parcelle carrée.
    \item $x_{ij} \in \{0,1\}$ : variable binaire associée à la parcelle $(i,j)$
    \begin{itemize}
        \item $x_{ij} = 1$ si la parcelle $(i,j)$ est \textbf{non coupée}
        \item $x_{ij} = 0$ si la parcelle $(i,j)$ est \textbf{coupée}
    \end{itemize}

    \item {$d_{ij} \ge 0$ : variable continue valant à l'optimum:
    
    \begin{itemize}
    \item $0$ si la parcelle $x_{ij}$ est coupée $(x_{ij} = 0)$ car le terme $|A_{ij}|(1-x_{ij})$ désactive la contrainte
    \item $\sum_{(k,l)\in A_{ij}} x_{kl}$ sinon soit le nombre de côtés de la parcelle $s_{ij}$ qui sont adjacents à des parcelles non coupées 
    \end{itemize} 
    
    Tout cela garantit bien que dans l'objectif les termes $(4x_{ij} - d_{ij})$ représentent le nombre de côtés étant une lisière pour les parcelles $(s_{ij})$ et donc 
    $g l \sum_{(i,j)\in M\times N} (4x_{ij} - d_{ij})$ est la population attendue de l'espèce 2}
    \item $t_{ij} \ge 0$ : population attendue de l'espèce $e_1$ dans la parcelle $(i,j)$ si elle est coupée.
    Donc $\sum_{(i,j)\in M\times N} t_{ij}(1 - x_{ij})$ est la population attendue le l'espèce 1
    \item $w_1, w_2$ : coefficients de pondération correspondant à l’importance relative des deux espèces dans la fonction objectif.
\end{itemize}

% ===========================
\section{Programme quadratique (P2)}
% ===========================
% → Tu détailleras ici la modélisation quadratique et son interprétation.

% ===========================
\section{Linéarisation du programme (P2)}
% ===========================
% → Présentation de la linéarisation et démonstration de la TU.

% ===========================
\section{Résolution numérique}
% ===========================
\subsection{Description des instances}
% → Décrire les données des instances tests
\subsection{Résultats expérimentaux}
% → Insérer tableaux et figures des résultats
\subsection{Comparaison des approches}
% → Comparaison en temps de calcul et qualité de solution

% ===========================
\section{Étude de sensibilité}
% ===========================
% → Étude de la contrainte sur le nombre minimal de parcelles non coupées.

% ===========================
\section{Conclusion}
% ===========================
% → Conclusion synthétique sur les approches, performances et perspectives.

% ===========================
\end{document}